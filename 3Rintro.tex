\documentclass{article}
\usepackage{3Rdefs}

\title{Understanding Reality Through Logic: Introducing 3R}
\date{}

\begin{document}

\maketitle

\subsection*{Language}

Human language operates in two domains: the real and the imaginary (or abstract). The real domain is rooted in our brain's visual and sensory systems, which model real objects amazingly well. The picture of reality animal brains form, based on neural modelling of light and other sensory inputs, is a quite remarkable product of evolution. Naming these objects enables person to person communication about real objects we observe. However, language also includes abstract words, terms which do not directly relate to sensory objects but may include opinions about sensory objects.

Imaginary words are great for fiction, but in the form of logic, they are also essential for logical modelling. You are not free to just invent logical words. Logical words (ideally mathematical symbols) must be consistently and unambiguously defined from the base definition of 0 defined as NOT 1. Logic is independent of reality but it is an amazingly powerful language due to its guaranteed true-false consistency. Logic defines \textbf{true} and \textbf{false}, but has very strict rules about how these abstract, logical words are formed and manipulated — they must not be political (axiomatic or belief-based).

A critical issue arises from illogical word combinations that blur the distinction between consistently defined logical words and words describing real objects. For instance, the term \textbf{real number} fails to understand both reality and logic. Numbers are not real, they are abstract, you will never find 1 or 0 lurking in the wild. Mathematicians and scientists must develop a clearer understanding of the distinction between consistently defined \textbf{logical} terms and how those terms are used to \textbf{model} reality. Reality is provably not constructed from logic, reality is scientifically modelled in the language of logic.

\subsection*{Logic and Modelling Uncertainty}

Logic, based on binary values (0 defined as NOT 1), eliminates ambiguity by disallowing contradictions. Logic is the foundation of mathematics and computing, providing a precise language for scientific models. However, reality, while scientifically modelled by 0 and 1, requires the definition of \qbit{} to incorporate uncertainty into models. \qbit{} represents 0 1 uncertainty, the super-position of discrete and continuous states, embodying both chance and causality. By incorporating \qbit{}, logical models can better reflect the complexities of reality.

\(\infty\) is a logically defined term, not a real object. \(\infty\) is no more real than 0 or 1. Traditional binary logic struggles to represent infinite sequences or continuous processes fully, leading to confusion among mathematicians and scientists. By defining \qbit{}, we create a logical framework which accepts uncertainty, allowing definition of \(\infty\) without falling into logical contradictions. The bicimal (binary equivalent of decimal) sequence 0.1, 0.01, 0.11, 0.001, ... shows we can construct a list of ever more precise values to count the \([0,1]\) continuum, but the counting will always be incomplete, always leaving \qbit{} uncertainty.

The important consideration is the balance of practicality and certainty for the task at hand. It is easy to both over-engineer and under-engineer models. The key is to understand the limits of certainty and to scientifically measure and declare uncertainty. Any model declaring itself certain is political not scientific.

\subsection*{Three Relativistic Dimensions (\iR{})}

\iR{} is the recognition that logical models of reality are optimised when they consider three relativistic dimensions. Unlike traditional models which rely on independent dimensions (length, width, height, time, etc.), \iR{} recognises these model variables as inter-connected features of a \iR{} relativistic model. This approach removes the inherent contradictions caused by the artificial edges of simple 0, 1 models. While many scientific models exceed human capabilities in specific areas, the concept of \iR{} is logically fundamental to the most complete and accurate models. Animal brains evolved to form three-dimensional neural models from sensory inputs for a reason. 

Logical models must however always acknowledge their inherent incompleteness and declare their limits of certainty. While models can achieve exceptionally high certainty—such as the three-dimensional neural models animal brains construct from sensory inputs—they are never perfect. \iR{}, however, offers the best possible logical modelling of reality. The relativistic \iR{} sphere, closely related to the 3D sphere is particularly interesting. Based on relativistic, not independent, dimensions, the \iR{} sphere offers a more comprehensive understanding of reality. If you have ever wondered how the universe can have no edge, and rejected the simplistic big-bang model, \iR{} offers a logical explanation, without resorting to the logical hacks of inflation, dark energy and dark matter.

\subsection*{Conclusion}

By understanding reality through the lens of \iR{} and \qbit{}, we can overcome the limitations posed by traditional binary logic. \iR{}, based on relativistic spheres and the super-position of chance and causality, provides a more accurate and intuitive way to describe reality. Science, while not perfect, provides the best possible models of reality, guiding us in making ever more logically informed decisions. With \iR{}, we can move beyond politics, and the artificial edges created by 0 and 1, embracing a more seamless and complete understanding of our world, whilst accepting uncertainty as an integral but quantifiable part of our journey.

For a more detailed and scientific exploration of these concepts, please refer to \href{https://www.researchgate.net/publication/379035220_3R_Scientific_Modelling}{3R Scientific Modelling} available on ResearchGate (3R.pdf from the Public Files drop down if you are reading this on ResearchGate). See also \href{https://www.researchgate.net/publication/375632465_0-Based_Logic}{0-Based Logic} which explains and fixes the error of ZFC (axiomatic mathematics).

\end{document}


